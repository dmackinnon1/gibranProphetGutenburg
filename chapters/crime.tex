
\lettrine{T}{hen} one of the judges of the city
stood forth and said, Speak to us of
\textit{Crime and Punishment}.

\medskip
And he answered, saying:

It is when your spirit goes wandering
upon the wind,

That you, alone and unguarded, commit
a wrong unto others and therefore unto
yourself.

And for that wrong committed must you
knock and wait a while unheeded at the
gate of the blessed.

Like the ocean is your god-self;

It remains for ever undefiled.

And like the ether it lifts but the
winged.

Even like the sun is your god-self;

It knows not the ways of the mole nor
seeks it the holes of the serpent.
But your god-self dwells not alone
in your being.

Much in you is still man, and much in
you is not yet man,

But a shapeless pigmy that walks asleep
in the mist searching for its own
awakening.

And of the man in you would I now speak.

For it is he and not your god-self nor
the pigmy in the mist, that knows crime
and the punishment of crime.



Oftentimes have I heard you speak of one
who commits a wrong as though he were
not one of you, but a stranger unto you
and an intruder upon your world.

But I say that even as the holy and the
righteous cannot rise beyond the highest
which is in each one of you,

So the wicked and the weak cannot fall
lower than the lowest which is in you
also.

And as a single leaf turns not yellow
but with the silent knowledge of the
whole tree, So the wrong-doer cannot
do wrong without the hidden will of you
all.

Like a procession you walk together
towards your god-self.

You are the way and the wayfarers.

And when one of you falls down he falls
for those behind him, a caution against
the stumbling stone.

Ay, and he falls for those ahead of him,
who though faster and surer of foot, yet
removed not the stumbling stone.

And this also, though the word lie heavy
upon your hearts:

The murdered is not unaccountable for
his own murder,

And the robbed is not blameless in being
robbed.

The righteous is not innocent of the
deeds of the wicked,

And the white-handed is not clean in the
doings of the felon.

Yea, the guilty is oftentimes the victim
of the injured,

And still more often the condemned is
the burden bearer for the guiltless
and unblamed.

You cannot separate the just from the
unjust and the good from the wicked;

For they stand together before the face
of the sun even as the black thread and
the white are woven together.

And when the black thread breaks, the
weaver shall look into the whole cloth,
and he shall examine the loom also.



If any of you would bring to judgment
the unfaithful wife,

Let him also weigh the heart of her
husband in scales, and measure his soul
with measurements.

And let him who would lash the offender
look unto the spirit of the offended.

And if any of you would punish in the
name of righteousness and lay the ax
unto the evil tree, let him see to its
roots;

And verily he will find the roots of the
good and the bad, the fruitful and the
fruitless, all entwined together in
the silent heart of the earth.

And you judges who would be just,

What judgment pronounce you upon him
who though honest in the flesh yet is a
thief in spirit?

What penalty lay you upon him who slays
in the flesh yet is himself slain in the
spirit?

And how prosecute you him who in action
is a deceiver and an oppressor,

Yet who also is aggrieved and outraged?



And how shall you punish those whose
remorse is already greater than their
misdeeds?

Is not remorse the justice which is
administered by that very law which you
would fain serve?

Yet you cannot lay remorse upon the
innocent nor lift it from the heart of
the guilty.

Unbidden shall it call in the night,
that men may wake and gaze upon
themselves. And you who would
understand justice, how shall you unless
you look upon all deeds in the fullness
of light?

Only then shall you know that the erect
and the fallen are but one man standing
in twilight between the night of his
pigmy-self and the day of his god-self,
And that the corner-stone of the temple
is not higher than the lowest stone in
its foundation.
