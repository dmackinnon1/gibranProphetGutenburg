\lettrine{T}{hen} said a rich man, Speak to us of
\textit{Giving}. 

\bigskip
And he answered:

You give but little when you give of
your possessions.

It is when you give of yourself that you
truly give.

For what are your possessions but things
you keep and guard for fear you may need
them tomorrow?

And tomorrow, what shall tomorrow bring
to the overprudent dog burying bones
in the trackless sand as he follows the
pilgrims to the holy city?

And what is fear of need but need
itself?

Is not dread of thirst when your well is
full, the thirst that is unquenchable?

There are those who give little of the
much which they have--and they give
it for recognition and their hidden
desire makes their gifts unwholesome.

And there are those who have little and
give it all.

These are the believers in life and
the bounty of life, and their coffer is
never empty.

There are those who give with joy, and
that joy is their reward.

And there are those who give with pain,
and that pain is their baptism.

And there are those who give and know
not pain in giving, nor do they seek
joy, nor give with mindfulness of
virtue;

They give as in yonder valley the myrtle
breathes its fragrance into space.

Through the hands of such as these God
speaks, and from behind their eyes He
smiles upon the earth.


It is well to give when asked, but it
is better to give unasked, through
understanding;

And to the open-handed the search for
one who shall receive is joy greater
than giving.

And is there aught you would withhold?

All you have shall some day be given;

Therefore give now, that the season
of giving may be yours and not your
inheritors’.

You often say, “I would give, but only
to the deserving.”

The trees in your orchard say not so,
nor the flocks in your pasture.

They give that they may live, for to
withhold is to perish.

Surely he who is worthy to receive his
days and his nights, is worthy of all
else from you.

And he who has deserved to drink from
the ocean of life deserves to fill his
cup from your little stream.

And what desert greater shall there be,
than that which lies in the courage
and the confidence, nay the charity, of
receiving?

And who are you that men should rend
their bosom and unveil their pride,
that you may see their worth naked and
their pride unabashed?

See first that you yourself deserve to
be a giver, and an instrument of giving.

For in truth it is life that gives unto
life--while you, who deem yourself a
giver, are but a witness.

And you receivers--and you are
all receivers--assume no weight of
gratitude, lest you lay a yoke upon
yourself and upon him who gives.

Rather rise together with the giver on
his gifts as on wings;

For to be overmindful of your debt, is
ito doubt his generosity who has the
free-hearted earth for mother, and God
for father.
